% document type
\documentclass[12pt]{article}

% packages
\usepackage[total={170mm,230mm}]{geometry}
\usepackage[utf8]{inputenc}
\usepackage[T2A]{fontenc}
\usepackage[russian]{babel}
\usepackage{graphicx}
\usepackage{amssymb}
\usepackage{amsfonts}
\usepackage{amsmath}
\usepackage{amsthm}
\usepackage{physics}
\usepackage{nicefrac}
\usepackage{cancel}
\usepackage{hyperref}
\usepackage{cmap}

% definitions
\DeclareMathOperator\arctanh{arctanh}
\DeclareMathOperator\arccosh{arccosh}
\DeclareMathOperator\const{const}
\newtheorem{definition}{Опредление}[section]
\newtheorem{theorem}{Теорема}[section]
\newtheorem{axiom}{Аксиома}[section]
\newtheorem{hypothesis}{Гипотеза}[section]

\title{Фотоэффект и измерение постоянной Планка}
\author{Краснощёкова Дарья \and Козлов Александр}

\begin{document}
	\maketitle

	\section{Теория} % (fold)
	\label{sec:theory}
	В настоящей работе будет исследоваться \emph{внешний фотоэффект}, который заключается в том, что тело эммитирует (испускает) во внешнее по отношению к нему пространство электроны под действием падающего света. При этом используется вакуумный фотоэлемент, имеющий два электрода: катод, испускающий электроны, и анод, их принимающий. Если на последний подать положительное напряжение (разность потенциалов между двумя электродами), то во внешней цепи будет течь ток, называемый \emph{фототоком}.

	\par Основные законы фотоэффекта, заключаются в следующем:
	\begin{enumerate}
		\item Величина фототока в режиме насыщения при фиксированном спектральном составе излучения прямо пропорциональна интенсивности падающего сета (\emph{закон Столетова}).

		\item Для каждого вещества существует длинноволновая граница фотоэффекта $\lambda_0$, за которой (при $\lambda > \lambda_0$) фотоэммисия не наблюдается.

		\item Максимальная кинетическая энергия электронов $W_{max}$ при фотоэффекте не зависит от интенсивности падающего света и прямопропорциональна частоте падающего света.
	\end{enumerate}

	Данные законы легко объясняются на основе квантовой теории света и электронной теории света. Свет обладает корпускулярно\--волновой дуальностью и может быть излучён либо поглащён отдельными квантами (\emph{фотонами}). Энергия одного фотона 
	$E = h \nu$, где $h$~\----~постоянная Планка, равная 

	\[
	h = 6.6260755(40)\cdot 10^{-34}\ \text{Дж}\cdot\text{с}.
	\]

	\par Закон Столетова объясняется следующим образом. Пускай на тело падает $N$ фотонов в единицу времени. С вероятностью $P$ каждый из них может "{}выбить"{} электрон. Так как каждый фотон действует на тело независимо, то всего электронов в единицу времени будет выбиваться $n = N\cdot P$. Интенсивность падающего света вычисляется по известному из электродинамики соотношению $I = \omega\cdot v$, где $\omega$~\----~плотность энергии электромагнитного излучения. Она пропорциональна 
	$\omega \propto N \cdot E$, где через $E$ обозначена энергия одного фотона. Таким образом, становится ясно, что фототок во внешней цепи 

	\begin{equation}
	 	I_\text{внеш} = en \propto N \propto \omega \propto I. 
	\end{equation} 

	\par Существование красной границы фотоэффекта объясняется тем, что электроны находится в металле в потенциальной яме, заполняя её до некоторого уровня. Чтобы вывести электрон из потенциальной ямы, а, следовательно, и из металла, нужно передать ему некоторую энергию. Наименьшая из таких энергий называется \emph{работой выхода} $A_{\text{вых}} =e\phi$.	Наибольшая энергия, которая может быть получена от фотона электроном, равна $h\nu$. Очевидно, что в случае $h\nu < A_{\text{вых}}$ фотоэффект не наблюдается.

	\par Из предыдущих соображений можно восстановить \emph{уравнение Эйнштейна}

	\begin{equation}\label{eq_einstain}
		W_{max} = h\nu - A_{\text{вых}}.
	\end{equation}

	Данное уравнение служит объяснением третьего закона фотоэффекта. 

	\par Для определения максимальной кинетической энергии электрона нужно подать на анод отрицательное напряжение $V$. Оно будет тормозить направляющиеся к аноду электроны. Все электроны, эммитированные с катода и облабающие энергией $W < eV$, не смогут добраться до анода. Уменьшая $V$ до напряжения, называемого \emph{запирающим}, можно добиться того, что ни один из электронов уже не будет обладать достаточно энергией для преодаления потенциального барьера между катодом и анодом. При этом 

	\begin{equation}
	 	W_{max} = e V_{\text{з}}.
	\end{equation} 

	Подставляем данное выражение в уравнение Эйнштейна (\ref{eq_einstain}) и получаем следующее линейное соотношение между запирающим напряжением и частотой падающего света:

	\begin{equation}
		V_{\text{з}} = \dfrac{h}{e}\nu - \varphi.
	\end{equation}
	% section теория (end)

	\section{Описание эксперимента} % (fold)
	\label{sec:exp_review}
	
	
	% section описание_эксперимента (end)
\end{document}